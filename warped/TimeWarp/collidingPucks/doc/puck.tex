\documentstyle[11pt]{report}

\newcommand{\version}{0.9}

\begin{document}

\title{
Colliding Pucks\\
A {\sc warped} simulation kernel application\\
(Documentation for version \version)}

\author{
\emph{Karthik Swaminathan}, \emph{Radharamananan Radhakrishnan}\\
and \emph{Philip A. Wilsey}\\
Computer Architecture Design Laboratory \\
Dept of ECECS, PO Box 210030 \\
Cincinnati, OH  45221--0030 \\
\tt{\{skarthik,ramanan,paw\}@ececs.uc.edu}
}
 
\date{}


\maketitle

\vspace*{4in}

\noindent
Copyright $\copyright$ 1997 The University of Cincinnati.  All
rights reserved.

\bigskip

\noindent
Published by the University of Cincinnati \\
Dept of ECECS, PO Box 210030 \\
Cincinnati, OH  45221--0030 USA

\bigskip

\noindent
Permission is granted to make and distribute verbatim copies of
this manual provided the copyright notice and this permission notice
are preserved on all copies.

\medskip
\noindent
Permission is granted to copy and distribute modified versions of this
manual under the conditions for verbatim copying, provided that the entire
resulting derived work is distributed under the terms of a permission
notice identical to this one.

\medskip
\noindent
Permission is granted to copy and distribute translations of this manual
into another language, under the above conditions for modified versions.

\newpage

\section*{Introduction}

This document describes the the Colliding Pucks application that is
distributed with the {\sc warped} project. The colliding pucks model has
been implemented as described by Beckman {\it et al} in their paper in the
SCS Multiconference on Distributed Simulation 1988. This is a simplified
model of a number of elastic pucks moving about on a flat surface (table)
and colliding among themselves and with the cushioned edges of the
table. The model is a highly simplified version of the real world and as
such is more synthetic than real. The user specifies the various details
needed for the simulation in a configuration file (described below). The
table is split into sectors. The number of sectors being specified by the
user. The movements of the pucks are modeled as messages while the sectors
form the simulation objects. This helps reduce the granularity of the
application as checks for collision have to be made only within the
sector. The sector is the only object in this model that is active during
the simulation. The other object just initializes the sector and sets up
the initial state of the pucks.

The program provided in the {\sc warped} distribution creates a PUCK
executable with the simulation objects (sectors) distributed among the
LPs. The number of simulation objects involved in the simulation is
specified by the user through the configuration file. The number of
objects allocated on each LP is an integer division of the total number of
objects and the number of LPs. The last LP will have more objects if the
numbers are not evenly divisible. Thus if we have 5 simulation objects (4
sectors and 1 Init object) simulating on 4 LPs, then the first 3 LPs will
have 1 object each while the last LP will have 2 objects.

The {\sc warped} simulator uses the MPI message protocol.  Thus, in order
to partition the simulation across several workstations, a file called
"procgroup" is used to identify the machines that will be used in the
simulation.  This file contains the addresses for each workstation, the
full path to the executable file for the simulation, and the number of
copies that each workstation will startup.  An example "procgroup" is
given below.

\begin{verbatim}
procgroup :

m1.ece.uc.edu 0 /home/username/warped/collidingPucks/PUCK
m2.ece.uc.edu 1 /home/username/warped/collidingPucks/PUCK
m3.ece.uc.edu 1 /home/username/warped/collidingPucks/PUCK
m4.ece.uc.edu 1 /home/username/warped/collidingPucks/PUCK
\end{verbatim}

In this example a four process simulation is being run on four different
workstations: m1, m2, m3, and m4 with each one executing one copy of the
executable "PUCK''.  The 0 first line of the "procgroup" file specifies
that machine m1 is starting the simulation.

NOTE: specifying that multiple copies should be started on a particular
machine does not work at this point in time, so all but one of the
procgroup entries should have a 1 in the second field.

\section*{Colliding Pucks Configuration}

The Colliding Pucks configuration file contains the simulation parameters in
the following order:
\begin{itemize}
\item Number of sectors per row M - the table is  a M X M plane
\item Width of each sector in grid units
\item Number of pucks
\item Number of LPs.

\end{itemize}

The configuration file should be named ``collidingPucks.config'' and must
be in the same directory as the executable. Here is an example of a
configuration file which will simulate with 16 sectors each of width 100
grid units, containing 100 pucks on 4 LPs.

\begin{verbatim}

Example collidingPucks.config:

4
100
100
4

\end{verbatim}

Note: In order to partition the simulation across workstations, a file
called ``procgroup'' must exist in the same directory as the executable.

In addition to these parameters one additional parameter that is required
by the simulation is the simulation time till which to simulate. This can
be modified by the user by changing the Simuntil value in ``main.cc''. For
example to run the simulation with the above configuration for a
simulation time of 50, the user must have in his main.cc the following:
\begin{verbatim}

LogicalProcess::SIMUNTIL = 50;
 
\end{verbatim}

The executable can be obtained using the Makefile provided with this
application. Certain changes might need to be made to the makefile for it
to work properly with your configurations and directory structure.

\newpage

\section*{Acknowledgments}
This research has been conducted with the participation of many
investigators.  While not an complete list, the following individuals
have made notable direct and/or indirect contributions to this effort
(in alphabetical order):
Perry Alexander,
Scott Bilik,
Harold Carter,
Dale E. Martin,
David A. Charley,
Girindra D. Sharma,
Praveen Chawla,
Debra A. Hensgen
John Hines,
Balakrishnan Kannikeswaran,
Venkatram Krishnaswamy,
Lantz Moore,
Avinash Palaniswamy,
John Penix,
Raghunandan Rajan.

We wish to express our sincerest gratitude for the time that you spent
reviewing and commenting on our designs.
\\
\\
This research was supported in part by the Advanced Research Projects
Agency, monitored by the Department of Justice under contract number
J--FBI--93--116.  In addition, we benefited greatly from the technical
support and guidance by the ARPA and DOJ program managers, notably: Bob
Parker, John Hoyt, and Lt. Col. John Toole.  Without this support and
interaction, the work documented in this report would not have been
possible.  Thank you.

\end{document}
