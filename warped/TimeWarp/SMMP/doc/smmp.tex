\documentclass[11pt]{report}

\usepackage{alltt,color,fullpage,psfig-dvips}

\newcommand{\version}{1.0}

\begin{document}

\title{
\textbf{SMMP}\\
A {\sc warped} simulation kernel application\\
(Documentation for version \version)}

\author{
\emph{Radharamanan Radhakrishnan} and \emph{Philip A.  Wilsey} \\
Computer Architecture Design Laboratory \\
Dept of ECECS, PO Box 210030 \\
Cincinnati, OH  45221--0030 \\
\texttt{warped@ececs.uc.edu}
}

\date{}

\maketitle

\vspace*{6in}

\noindent
Copyright $\copyright$ 1995--1999 The University of Cincinnati.  All
rights reserved.  

\bigskip

\noindent
Published by the University of Cincinnati \\
Dept of ECECS, PO Box 210030 \\
Cincinnati, OH  45221--0030 USA 

\bigskip

\noindent
Permission is granted to make and distribute verbatim copies of
this manual provided the copyright notice and this permission notice
are preserved on all copies.

\medskip
\noindent
Permission is granted to copy and distribute modified versions of this
manual under the conditions for verbatim copying, provided that the entire
resulting derived work is distributed under the terms of a permission
notice identical to this one.

\medskip
\noindent
Permission is granted to copy and distribute translations of this manual
into another language, under the above conditions for modified versions.

\newpage

\chapter{Introduction}
Included in the {\sc warped} distribution is a simulation library for
building queuing models. This library is called \texttt{KUE}. In addition
to the library, the distribution includes several instances of use of
KUE. SMMP is one example of a model constructed with the aid of the KUE
queuing model library.
\\
\\
This document describes the SMMP symmetric multiprocessor application
model. The SMMP application models a shared memory multiprocessor. Each
processor is assumed to have a local cache with access to a common global
memory (The model is somewhat contrived in that requests to the memory is
not serialized -- {\it i.e.,} main memory can have multiple requests
pending at any given moment). The model is generated by a C++ program
which lets the user adjust several model parameters. The C++ program ({\it
MemGen.cc}) also partitions the model to take advantage of the fast
intra-processor communication. The next section deals with the {\it
MemGen.cc} program.

\section{MemGen.cc - Model Generator program}

The model generator program lets the user adjust the following parameters
before writing out the model:
\begin{itemize}
\item (a) Number of processors or caches to simulate - This is the number
of SMMP processors that are required to be simulated. Each SMMP processor
is represented by a set of queuing objects that perform the task of
symmetric multiprocessor. In our model, a processor along with its cache
memory is modeled with six queuing objects (Source, Join, Queue, Fork,
Server and another Join). Figure~\ref{fig:smmp} illustrates the complete
set of queuing objects that are required for modeling a 8 processor
symmetric multiprocessor.

\item (b) A Sequential or a Parallel Simulation - The {\sc warped}
simulation kernel can be deployed as a parallel simulation kernel as well
as a sequential kernel.

\item (c) Number of physical processors (or Logical Processors) - This is
the number of processors that the user wants to use on the host machine
for a parallel simulation. The generator program requires that the number
of processors or caches ({\it i.e.,} option (a)) be an integral multiple
of the number of physical processors. This is required to create correct
partitions. 

\item (d) Speed of cache - This is a model parameter that needs to be set
by the user. We assume that the speed of the cache is an integer quantity
which is smaller than the speed of the main memory. A typical value for
the speed of the cache would be 10, whereas the speed of the main memory
can be 100, implying that the cache is 10 times faster than the main
memory. The main objective here is to model the latencies of the cache and
the main memory.

\item (e) Cache Hit ratio - again this is another model parameter that
determines what percentage of the requests are sent to the cache memory
instead of the main memory. Obviously, the lower this number is, the more
requests that are sent out to the main memory ( {\it i.e.,} more cache
misses). 

\item (f) Speed of main memory - Basically the latency of the main memory
represented as a multiple of the cache speed. Again we assume here that
this is an integer quantity.

\item (g) Number of requests per processor - This is the number of memory
requests each processor with generate during the lifetime of the
simulation.

\item (h) Name of file to generate - The generator will write out the
model to the file name specified by the user. By default, the generator
program will write out the model to {\it main.cc}. 
\end{itemize}

\begin{figure}
\centerline{\psfig{figure=figures/Smmp_ex_fig.ps,width=6in,silent=}}
\caption{A 8 processor SMMP model}\label{fig:smmp}
\end{figure}

\chapter{User Guidelines}
Now that the user is familiar with generating the SMMP model, this section
discusses the procedure of simulating the generated
model. Figure~\ref{fig:smmp} illustrates the set of queuing objects that
are required to construct a 8 processor SMMP model. The model included in
the distribution (namely main.cc) is a 2 processor SMMP model that is
intendend to be simulated on a 2 processor machine. The {\sc warped}
simulator uses the MPI message protocol.  Thus, in order to partition the
simulation across several workstations, a file called "procgroup" is used
to identify the machines that will be used in the simulation.  This file
contains the addresses for each workstation, the full path to the
executable file for the simulation, and the number of copies that each
workstation will startup.  An example "procgroup" is given below.

\begin{verbatim}
procgroup :

m1.ece.uc.edu 0 /home/username/warped/SMMP/SMMP_EXEC
m2.ece.uc.edu 1 /home/username/warped/SMMP/SMMP_EXEC
m3.ece.uc.edu 1 /home/username/warped/SMMP/SMMP_EXEC
m4.ece.uc.edu 1 /home/username/warped/SMMP/SMMP_EXEC
\end{verbatim}

In this example a four process simulation is being run on four different
workstations: m1, m2, m3, and m4 with each one executing one copy of the
executable "SMMP\_EXEC''.  The 0 first line of the "procgroup" file
specifies that machine m1 is starting the simulation.

NOTE: specifying that multiple copies should be started on a particular
machine does not work at this point in time, so all but one of the
procgroup entries should have a 1 in the second field.



\chapter{Acknowledgments}
This research has been conducted with the participation of many
investigators.  While not an complete list, the following individuals
have made notable direct and/or indirect contributions to this effort
(in alphabetical order):
Perry Alexander,
Scott Bilik,
Harold Carter,
Dale E. Martin,
David A. Charley,
Girindra D. Sharma,
Praveen Chawla,
Debra A. Hensgen
John Hines,
Balakrishnan Kannikeswaran,
Venkatram Krishnaswamy,
Lantz Moore,
Avinash Palaniswamy,
John Penix,
Radharamanan Radhakrishnan,
Raghunandan Rajan, 
Dhananjai Madhava Rao, and
Christopher Young.
We wish to express our sincerest gratitude for the time that you spent
reviewing and commenting on our designs.
\\
\\
This research was supported in part by the Advanced Research Projects
Agency, monitored by the Department of Justice under contract number
J--FBI--93--116.  In addition, we benefited greatly from the technical
support and guidance by the ARPA and DOJ program managers, notably: Bob
Parker, John Hoyt, and Lt. Col. John Toole.  Without this support and
interaction, the work documented in this report would not have been
possible.  Thank you.

\end{document}
