\documentclass[11pt]{report}

\usepackage{alltt,color,fullpage,psfig-dvips}

\newcommand{\version}{1.0}

\begin{document}

\title{
\textbf{RAID}\\
A {\sc warped} simulation kernel application\\
(Documentation for version \version)}

\author{
\emph{Christopher H. Young}, \emph{Radharamanan Radhakrishnan} and \emph{Philip A.  Wilsey} \\
Computer Architecture Design Laboratory \\
Dept of ECECS, PO Box 210030 \\
Cincinnati, OH  45221--0030 \\
\texttt{warped@ececs.uc.edu}
}

\date{}

\maketitle

\vspace*{6in}

\noindent
Copyright $\copyright$ 1995--1999 The University of Cincinnati.  All
rights reserved.  

\bigskip

\noindent
Published by the University of Cincinnati \\
Dept of ECECS, PO Box 210030 \\
Cincinnati, OH  45221--0030 USA 

\bigskip

\noindent
Permission is granted to make and distribute verbatim copies of
this manual provided the copyright notice and this permission notice
are preserved on all copies.

\medskip
\noindent
Permission is granted to copy and distribute modified versions of this
manual under the conditions for verbatim copying, provided that the entire
resulting derived work is distributed under the terms of a permission
notice identical to this one.

\medskip
\noindent
Permission is granted to copy and distribute translations of this manual
into another language, under the above conditions for modified versions.

\newpage

\chapter{Introduction}

This document describes the RAID-5 Disk Array application model that is
distributed with the {\sc warped} project.  RAID Disk Arrays are a method
that provides a vast array of storage by using a Redundant Array of
Inexpensive Disks to provide storage with a higher I/O performance than
several large expensive disks.  This application incorporates a flexible
model of a RAID Disk Array using any one of three possible disk drives.
This application can be configured in various sizes of disk arrays to be
studied.  More precisely, a C program generates code to model various
configurations of disk arrays and request generators. 

The RAID simulation model models the interactions of several processes
with a level 5 RAID Disk Array composed of one of 3 possible disks,
namely:
\begin{itemize}
\item
a Lighting Disk modeled after the IBM 0661 3.25 inch SCSI disk drive
\item
a Fujitsu disk M2652H/S 5.25 inch 1.8GB SCSI, and
\item
a future disk that is an extrapolation from the Fujitsu disk's
parameters. 
\end{itemize}
Explanations of these disks and their parameters can be found in the
SIGMETRICS '93 and ASPLOS '91 papers by Lee and Katz.

The program provided in the {\sc warped} distribution creates a RAID
simulation with the simulation objects distributed uniformly among the
LPs. This program creates and partitions an arbitrarily sized simulation
of processes, forks, and disks. A configuration file (described below)
is used to control the generation of simulation objects and the number
of requests issued by the source processes to the disks. After reading
the configuration file, each LP will dynamically allocate each
object. The number of each type of object allocated on an LP will be an
integer division of the the total number and the number of LPs in the
simulation. The last LP will contain more objects of that type if the
numbers are not evenly divisible. Thus, if 5 source processes are to be
simulated on 4 LPs then each LP will have 1 source process and the last
LP will have 2 source processes.

The {\sc warped} simulator uses the MPI message protocol.  Thus, in order
to partition the simulation across several workstations, a file called
"procgroup" is used to identify the machines that will be used in the
simulation.  This file contains the addresses for each workstation, the
full path to the executable file for the simulation, and the number of
copies that each workstation will startup.  An example "procgroup" is
given below.

\begin{verbatim}
procgroup :

m1.ece.uc.edu 0 /home/username/warped/RAID/RAID_EXEC
m2.ece.uc.edu 1 /home/username/warped/RAID/RAID_EXEC
m3.ece.uc.edu 1 /home/username/warped/RAID/RAID_EXEC
m4.ece.uc.edu 1 /home/username/warped/RAID/RAID_EXEC
\end{verbatim}

In this example a four process simulation is being run on four different
workstations: m1, m2, m3, and m4 with each one executing one copy of the
executable "RAID\_EXEC".  The 0 first line of the "procgroup" file
specifies that machine m1 is starting the simulation.

NOTE: specifying that multiple copies should be started on a particular
machine does not work at this point in time, so all but one of the
procgroup entries should have a 1 in the second field.

\section{RAID Config}

The RAID configuration file contains the parameters for the RAID
simulation in the following order:

\begin{itemize}
\item
the number of source processes to generate the requests,
\item
the number of fork processes to route the request, 
\item
the number of disks in the disk array,
\item
the number of tokens to be generated by each process, and
\item
the number of LPs.
\end{itemize}

The RAID configuration file should be named "RAID.config", and there
should be a link, called RAID.config, to this file in the users top
level directory.  Here is an example of a configuration file which will
simulate 30 processes each sending 10 tokens, 4 fork objects routing the
requests, which route the requests to 10 disks, and will simulate on
4 LPs.

\begin{verbatim}
Example RAID.config:

30
4
10
10
4
\end{verbatim}

Note: In order to partition the simulation across workstations, a file
called "procgroup" must exist the same directory as the executable
file.

\chapter{RAID Processes}

A process generates requests for the RAID Array.  Each request is for a
stripe of random access type (r/w), length, and location.  After sending
the message, the process is assumed to be suspended until all of the
request messages (including parity requests) are received.  After all
messages have been received then the process generates a new request
after some random delay.

The process requires the following information to be passed into its
constructor:

\begin{itemize}
\item
A unique identifier.

\item
A name string. 

\item
The id of a fork to process its requests.

\item
The number of disks in the simulation.

\item
The type of disks in the RAID array.

\item
The number of requests it should generate.

\end{itemize}

The process also has a function "setDistribution" which sets the
distribution for the time delay between the completion of a request and
the next request.

\chapter{RAID Forks}

A fork takes a request from a source process and breaks it up into
individual requests for each disk of the array, calculates the number of
parity requests that are needed, and forwards these requests to the
disks.  A fork will first calculate the number of parity requests that
are needed by calculating the parity destination for each request.
Duplicate requests are found by comparing the parity request for the
current message to the request for the last message.

The current model is for a RAID Level 5 Disk Array with left--symmetric
parity placement.  RAID Level 5 distributes the parity information
across the disk array, so no disk contains the entire parity
information.  The left--Symmetric parity placement policy places the
parity information for the n parity stripes units in a round--robin
fashion starting from the last disk.  An example for 4 disks is shown
below.

\begin{verbatim}
| 0  | 1  | 2  | P  |
| 3  | 4  | P  | 5  |
| 6  | P  | 7  | 8  |
| P  | 9  | 10 | 11 |
| 12 | 13 | 14 | P  |
\end{verbatim}

Each number represents a stripe unit, and P represents parity
information.  Each row represents a stripe of data along with the parity
information.  Each column represents the data on a single disk.

The fork requires the following information to be passed into its
constructor:
\begin{itemize}

\item
A unique identifier.

\item
A name string. 

\item
The number of disks in the simulation.

\item
The number processes in the simulation.

\item
The id of the first disk in the simulation.

\end{itemize}

NOTE: The last two items are used determine by the fork to determine
disk identifiers for routing the requests. Figure~\ref{fig:raid_fig}
illustrates a typical RAID example. The example has 16 sources routing
requests through a single fork to 10 different disks.

\begin{figure}
\centerline{\psfig{figure=figures/RAID_fig.ps,width=6in,silent=}}
\caption{An example RAID model}\label{fig:raid_fig}
\end{figure}

\chapter{RAID Disks}

A Disk takes the request and seeks to the proper cylinder using a model
of the seek time calculated found in the papers by Lee and Katz.  The
current model ignores the sector information and assumes that half the
time of a single rotation is needed to position for the sector to appear
under the head.

The disk requires the following information to be passed into its
constructor:

\begin{itemize}

\item
A unique identifier.

\item
A name string. 

\item
The type of disk to be simulated.

\end{itemize}


\chapter*{Acknowledgements}
This research has been conducted with the participation of many
investigators.  While not an complete list, the following individuals
have made notable direct and/or indirect contributions to this effort
(in alphabetical order):
Perry Alexander,
Scott Bilik,
Harold Carter,
Dale E. Martin,
David A. Charley,
Girindra D. Sharma,
Praveen Chawla,
Debra A. Hensgen
John Hines,
Balakrishnan Kannikeswaran,
Venkatram Krishnaswamy,
Lantz Moore,
Avinash Palaniswamy,
John Penix,
Radharamanan Radhakrishnan,
Dhananjai Madhava Rao, and
Raghunandan Rajan.

We wish to express our sincerest gratitude for the time that you spent
reviewing and commenting on our designs.
\\
\\
This research was supported in part by the Advanced Research Projects
Agency, monitored by the Department of Justice under contract number
J--FBI--93--116.  In addition, we benefited greatly from the technical
support and guidance by the ARPA and DOJ program managers, notably: Bob
Parker, John Hoyt, and Lt. Col. John Toole.  Without this support and
interaction, the work documented in this report would not have been
possible.  Thank you.

\end{document}





